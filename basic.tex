\vspace{5 mm}  %to enter vertical space
\subsection*{Entities}
Entities are graphical objects in a CAD system. Typical entities which are supported by most CAD systems are: points, lines and circular and elliptical arcs. More complex and CAD specific entities include polylines, texts, dimensioning, hatches and splines.
\subsection*{Layers}
A basic concept in computer aided drafting is the use of layers to organize a drawing. Every entity in a drawing is on exactly one layer and a layer can contain lots of entities. Typically entities with a common 'function' or common attributes are put on the same layer. E.g. you might want to put all axes in a drawing on a layer named `axes'. Layers can have their own attributes also (colour, line width, line style etc...). Each entity can have its own attributes or have its attributes defined by the layer it is placed on. In the latter case for example you can change the colour of all the entities on the ``axes'' layer by setting the colour (red for example) for the layer `axes'.
\subsection*{Blocks}
A Block is a group of entities. Blocks can be inserted into the same drawing more than once with different Attributes and at different locations and at a different scale and rotation angle . Such an instance of a block is usually called an Insert. Inserts have attributes just like entities and layers. An Entity that is part of an Insert can have its own attributes or share the attributes of the Insert. Once created, Inserts are still linked to the Block they instantiate. The power of inserts is, that you can modify the Block once and all Inserts will be updated accordingly.
\subsection*{Coordinate System}
In order to get the best out of LibreCAD it is wise to have a good understanding of the coordinate system and how coordinates work. Everything that you draw in LibreCAD will be exact and precise and will be placed there accurately based on the X,Y coordinate system.
The absolute origin or Zero point in your drawing is where the X and Y axes cross each other (represented by a Red cross), every entity you draw is located in relation to this origin.\\
In LibreCAD there is also the option to set the Relative Zero Point (small red circle).This Relative zero point can be temporarily set to a new location in a drawing so that all subsequent X and Y coordinates of entities drawn or blocks placed for example will be relative to this newly set Relative Zero Point.\\
In LibreCAD's 2D coordinate system all X units are measured horizontally and all Y units are measured vertically. Coordinates can also be shown as `Positive' (+) or `Negative'(-) values.\\
Basically there are two types of Coordinates \textbf{Cartesian and Polar}.
The \textbf{Cartesian coordinate system} is generally the standard system used in most CAD programs. A specific point in a drawing is located by exact distances from both the X and Y axes - for example a point in a drawing could be 60,45 (note the comma -, separates the two numbers).\\
The \textbf{Polar coordinate system} uses one distance and one angle to define a point in a drawing -for example a point in a drawing could be 50 $<$ 45, so 50 units long and at an angle of 45 degrees (note the $<$ sign is used for the angle).\\\\
In LibrecAD lines,points, Arcs, Polylines, Circles and many more entities can be drawn and placed in a drawing using either \textbf{Absolute or Relative coordinate} input.\\
\textbf{Absolute coordinates} - using this method,coordinate points are entered in direct relation to the Origin 0,0. To do this in LibreCAD just enter in the exact point e.g. 60,45.\\
\textbf{Relative coordinates} - using this method, coordinate points are entered in relation to the previous point entered (not the origin), so for example - if your first point is 20,45, to then enter your next point `relative' to this - you would use the `@' symbol - e.g @50,50 would then enter the second point 50 units horizontally along the x axis and 50 units vertically along the Y axis to give this second point relative to your last point (20,45).
\subsection*{Dimensioning}
Required sizes of objects within a drawing are conveyed through the use of dimensions. Dimension `distances' may be shown with either of two standardized forms of dimension - \textbf{Linear and ordinate}.\\
\textbf{Linear dimensions} use two parallel lines - called ``extension lines,'' which are spaced at the `required' distance between two given points.A line perpendicular to these extension lines is called a ``dimension line'', with arrows at its endpoints. The numerical indication of the distance is placed at the midpoint of the dimension line, adjacent to it or in a gap provided for it!\\
\textbf{Ordinate dimensions} use one horizontal and one vertical extension line to establish an origin for the entire view. The origin is identified with 0,0 placed at the ends of these extension lines. Distances along the x and y axes to other points on a drawing are indicated using additional extension lines with numerical information placed appropriately.
